\documentclass{article}
\usepackage{graphicx} % Required for inserting images
\usepackage[utf8]{inputenc} %кодировки
\usepackage[T2A]{fontenc} %кодировки
\usepackage[english,russian]{babel} % поддержка языков
\usepackage{txfonts}
\usepackage{amsmath}
\usepackage{amssymb}
\usepackage{mathtools}

\title{Курс комплексные числа}




\begin{document}

    \begin{center}
    \textbf{\Large Комплексные числа}
    \end{center}
    Для изучения комплексных чисел сначала стоит вспомнить, какие числа вообще существуют.
    
    \begin{itemize}
    
        \item {Натуральные числа}
            \[
            \mathbb{N}=\{1,2,3,\ldots,\infty\}
            \]
            Все числа, используемые для счета. С их помощью можно решать простейшие уравнения вида 
            \[
            x-3 = 7
            \]
    
    
        \vspace{1cm}   
        \item {Целые числа}
            \[
            \mathbb{Z}=\{0,\pm1,\pm2,\pm3,\ldots,\infty\}
            \]
            Рассматриваются, как дополнение множества натуральных чисел, путем добавления всех целых отрицательных чисел и нуля.
            Теперь можно решать уравнения, решения которых в поле натуральных чисел не существует, а в поле целых существует
             \[
            x+3 = 1
            \]
    
        \vspace{1cm}
        \item {Рациональные числа}
            \[ 
            \mathbb{Q}=\{\frac{m}{n} : m\in \mathbb{Z}, n\in \mathbb{N}\}
            \]
            Проще говоря, все числа, представимые в виде обыкновенной дроби.
    
        \vspace{1cm}
        \item {Иррациональные числа}
            $$
            x = a_0{,}a_1a_2a_3\ldots \in \mathbb{I},
            $$
            где:
            \begin{itemize}
              \item $a_0 \in \mathbb{Z}$ — целая часть;
              \item $a_1, a_2, a_3, \ldots \in \{0, 1, \ldots, 9\}$ — цифры после запятой;
              \item не существует таких $k \in \mathbb{N}$ и $p \in \mathbb{N}$, что для всех $i \geq k$ выполняется $a_i = a_{i+p}$.
            \end{itemize} 
            Примеры: \sqrt{2}, \pi, e, \ln 2
    
        \vspace{1cm}
        \item{Вещественные чиcла}
            \[ 
            \mathbb{R}=\mathbb{Q} \cup  \mathbb{I}
            \]
    
            Вещественные числа мы можем каким то образом отмерить на числовой прямой. Корень из двух например это диагональ квадрата со стороной 1.
    
        \vspace{1cm}
        \item {Комплексные числа}
        ?
    \end{itemize}

    \vspace{3cm}
    Казалось бы, зачем нам комплексные числа, если мы и так спокойно решаем все задачи, которые стоят перед нами. Продолжалась такая ситуация до 16 века.
    
    Идея использования комплексных чисел зародилась в XVI веке в связи с решением кубических уравнений. Впервые мнимые величины были упомянуты в труде Джироламо Кардано «Великое искусство, или об алгебраических правилах» (1545 год). Кардано рассматривал задачу на нахождение двух чисел, сумма которых равна 10, а произведение — 40. В решении получилось выражение с квадратным корнем из отрицательного числа:


    Рафаэль Бомбелли (итальянский математик и инженер) в книге «Алгебра» (1572 год) впервые ввёл правила арифметических операций над комплексными числами, включая сложение, вычитание и умножение, а также извлечение кубических корней. Он показал, что при решении некоторых кубических уравнений корни из отрицательных величин могут взаимно уничтожаться, если слагаемые являются взаимно сопряжёнными. Бомбелли относился к комплексным числам как к полезной вспомогательной конструкции. 



    Термин «мнимые числа» ввёл в 1637 году Рене Декарт в работе «Геометрия». В 1777 году Леонард Эйлер предложил использовать букву i для обозначения мнимой единицы (от латинского imaginarius —  «мнимый»). Эйлер также распространил на комплексную область стандартные функции, включая логарифм, и высказал мысль, что в системе комплексных чисел любой многочлен имеет корень (основная теорема алгебры). 

    Карл Фридрих Гаусс в 1831 году ввёл в широкое употребление термин «комплексные числа» (от латинского complexus —  «связь, сочетание»). Он дал геометрическую интерпретацию комплексных чисел, представив их в виде точек на плоскости (так называемая плоскость Гаусса). Гаусс также построил арифметическую модель комплексных чисел как пар вещественных чисел, что доказало непротиворечивость их свойств. 



    \vspace{4cm}
    Для лучшего понимания процесса пройдем по пути Карадно, который столкнулся с комплексными числами при решении кубического уравнения. Для начала докажем формулу Кардано.

    \section*{Формулировка}
    Пусть дано приведённое кубическое уравнение:
    \begin{equation}
        x^3  -3px - 2q = 0 
    \end{equation}
    

    где $p, q \in \mathbb{R}$.
        
    Тогда его корни находятся по \textbf{формуле Кардано}:
    \[
    x = \sqrt[3]{q + \sqrt{q^2 - p^3}} + \sqrt[3]{q - \sqrt{q^3 - p^3}}
    \]

    \section*{Доказательство:}
    Для начала стоит сказать, что любое кубическое уравнение можно свести к приведенной форме, данной в формулировке, путем введения замены и выделения полного куба. (напиши пару строчек, но строго и полностью доказывать это не буду)

    Пусть $x$ - решение приведенного кубического уравнения (1). Представим его как сумму двух чисел $\alpha$ и $\beta$.
    Тогда $x=\alpha + \beta.$
    \[
    x^3 = \alpha^3 + \beta^3 + 3\alpha\beta(\alpha + \beta) = \alpha^3 + \beta^3+ 3\alpha\beta x 
    \]
    Тогда уравнение (1) препишется так:
    \begin{equation}
        \alpha^3 + \beta^3+ 3\alpha\beta x - 3px-2q=0
    \end{equation}
     
    
    
    



    Перегруппируем слагаемые в (2):
    \begin{equation}
    (\alpha^3 + \beta^3 -2q) +3x(\alpha\beta-p)  = 0
    \end{equation}
    
    Чтобы уравенние (3) имело решение, должны выполнятся следующие условия:
    
    \begin{equation}
        \begin{cases}
            \alpha\beta = p \\
            \alpha^3 + \beta^3 = 2q
        \end{cases}
    \end{equation}
    
    \begin{equation}
        \begin{cases}
            \alpha^3\beta^3 = p^3 \\
            \alpha^3 + \beta^3 = 2q
        \end{cases}
    \end{equation}
    
    Вспомним теорему Виета. Уравнение вида $x^2-px+q$ имеет решения $x_1$, $x_2$ такие что:
    
    \begin{equation}
        \begin{cases}
            x_1+x_2 = -p \\
            x_1x_2 = q
        \end{cases}
    \end{equation}
    
    Тогда пусть $(y-\alpha^3)(y-\beta^3)=0$ есть некоторое уравенние с решениями $\alpha^3$ и $\beta^3$. К ним можно применить теорему Виета (как бы в обратную сторону. из имеющихся корней конструируем квадратное уравнение)
    
    \begin{equation}
        \begin{cases}
            \alpha^3 + \beta^3 = 2q \\
            \alpha^3\beta^3 = p^3 
        \end{cases}
    \end{equation}
    
    Осталось сконструировать из этой системы уравение
    \begin{equation}
        y^2-2qy+p^3=0
    \end{equation}
    Таким образом мы свели кубическое уравнение к квадратному (его корни $\alpha^3$ и $\beta^3$)
    
    
    
    
    \begin{equation}
        D = 4q^2 - 4p^3 = 4(q^2 - p^3)
    \end{equation}
    Корни квадратного уравнения:
    \begin{equation}
        y_{1,2} = q \pm \sqrt{q^2 - p^3}
    \end{equation}
    Пусть:
    \begin{equation}
        \alpha^3 = q + \sqrt{q^2 - p^3}, \quad
        \beta^3 = q - \sqrt{q^2 - p^3}
    \end{equation}
    
    
    Так как $x = \alpha + \beta$, то:
    \begin{equation}
        x = \sqrt[3]{q + \sqrt{q^2 - p^3}} + \sqrt[3]{q - \sqrt{q^2 - p^3}}
    \end{equation}



    

    \vspace{4cm}
    Теперь, когда мы доказали формулу Кардано, мы можем ей пользоваться. Давйте с помощью нее решим уравнение.
    Пусть дано следующее уравнение
    \[
    (x-1)(x-2)(x+3)=0,
    \]
    очевидно, что у оно имеет три корня: $x=1, 2, -3$. Запомним это. Раскроем скобки и превратим уравнение в полином 3-ей степени
    \[
    x^3-7x+6=0.
    \]
    Для решения полученного уравнения воспользуемся формулой Кардано
    \[
    x^3+3px-2q=0
    \]
    \[
    x=\sqrt[3]{q+\sqrt{q^2-p^3}}+\sqrt[3]{q-\sqrt{q^2-p^3}}
    \]
    В нашем случае $p=\frac{7}{3}$, $q=-3$
    \[
    x=\sqrt[3]{-3+\sqrt{9-\frac{343}{27}}}+\sqrt[3]{-3-\sqrt{9-\frac{343}{27}}}=
    \sqrt[3]{-3+\sqrt{\frac{-100}{27}}}+\sqrt[3]{-3-\sqrt{\frac{-100}{27}}}
    \]
    Видим корень из -100. Получается чтобы решить уравнение третьей степени, корни которого вполне вещественны, нам нужно зайти в комплексные числа. Именно для этого впервые и понадобились комплексные числа и новый аппарат для работы с ними.
    \vspace{1cm}
    
    $\sqrt{-100} = \sqrt{-1}*\sqrt{100} = i*10 = 10i$
    
    Таким образом $i=\sqrt{-1}$
    



\end{document}
