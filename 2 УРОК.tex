\documentclass{article}
\usepackage{graphicx} % Required for inserting images
\usepackage[utf8]{inputenc} %кодировки
\usepackage[T2A]{fontenc} %кодировки
\usepackage[english,russian]{babel} % поддержка языков
\usepackage{txfonts}

\usepackage{tikz}
\usepackage{pgfplots}
\pgfplotsset{compat=1.18}

\usepackage{amsmath}
\usepackage{amssymb}
\usepackage{mathtools}

\title{Курс комплексные числа}

    


\begin{document}
    Теперь мы знаем, что существует число $i \in \mathbf{C}$. Давайте же рассмотрим его подробнее.
    \vspace{1cm}
    
    \textbf{\small (Далее я расскажу про комплексную плоскость. Расставлю числа на осях. Расположу на ней пару чисел ($3+2i$ то же самое, что перенос точки (0,0) сначала на три еденицы вправо, затем на 2 еденицы вверх)}

    \vspace{1cm}

    Любое комплексное число $z=x+iy$ можно найти на комплексной плоскости. Далее вещественной частью числа $z$ будем называть число $Re\hspace{3px}z$ ($Re$-реальная), а мнимой частью число $Im\hspace{3px}z$ ($Im$-imagine)
    \[
    Re\hspace{3px}z = x
    \]
    \[
    Im\hspace{3px}z = y
    \]

    \vspace{3cm}


    \begin{tikzpicture}[scale=1.2]
        % Координатная плоскость
        \draw[->, line width=1pt] (-4,0) -- (4,0) node[below] {Действительная ось $\mathbb{R}$};
        \draw[->, line width=1pt] (0,-3) -- (0,3) node[left] {Мнимая ось $i\mathbb{R}$};
        
        % Деления
        \foreach \x in {-3,-2,-1,1,2,3}
            \draw (\x,0.1) -- (\x,-0.1) node[below] {$\x$};
        \foreach \y in {-2,-1,1,2}
            \draw (0.1,\y) -- (-0.1,\y) node[left] {$\y i$};
        
        % Точки
        \fill[red] (2,1.5) circle (3pt) node[above right] {$z_1=2+1.5i$};
        \fill[blue] (-1,-2) circle (3pt) node[below left] {$z_2=-1-2i$};
        \fill[green] (0,2) circle (3pt) node[above left] {$z_3=2i$};
        \fill[orange] (3,0) circle (3pt) node[above right] {$z_4=3$};
        
        % Векторы
        \draw[->, red, thick] (0,0) -- (2,1.5);
        \draw[->, blue, thick] (0,0) -- (-1,-2);
        \draw[->, green, thick] (0,0) -- (0,2);
        \draw[->, orange, thick] (0,0) -- (3,0);
        
        % Комплексное сопряжение
        \fill[purple] (1,-1.5) circle (3pt) node[below right] {$\overline{z}$};
        \fill[purple] (1,1.5) circle (3pt) node[above right] {$z$};
        \draw[dashed, purple] (1,1.5) -- (1,-1.5);
    \end{tikzpicture}

    \vspace{3cm}
    Далее последует короткое задание на определение расположение компексных чисел на плоскости.  


    
    \vspace{3cm}

    \textbf{Вопрос в аудиторию}
    
    "Может ли комплексных чисел быть больше, чем точек на плоскости?"

    "Иначе говоря, могут ли два разных комплексных числа распологаться в одной точке?
    
    "Нет, не могут, каждое число уникально, вот доказательство. это понадобиться чтобы правильно ввести операции над комплексными числами"

    \begin{figure}[h]
        \centering
        \includegraphics[width=0.5\linewidth]{фото для урока 2.jpg}
        \caption{позже обяз доделаю и затехаю, ща лень}
        \label{fig:placeholder}
    \end{figure}
    
    
    
    \vspace{4cm}

    Таким обрзазом мы определили множество чисел, с которыми будем работать, это все числа вида $x+iy\in \mathbb{C}$ ($\mathbb{C}$-множество комплексных чисел). Далее нужно определить стандартные оперции над этим множеством. (порадить поле (понятие поля не будет вводиться))


    \vspace{1cm}
    \section*{Разберемся со сложением $(\mathbb{C,+})$}

    Простой пример. Возмем два компексных числа $(z=x+iy)$ и $(w=a+ib)$ и попробуем их сложить как обыкновенные вещественные числа. 
    \[
    z+w = x+iy+a+ib =[group]=(x+a)+i(b+y)
    \]
    Как мы видим получилось компекносе число. Так и рабоатет сложение в поле коплексных чисел. Складываются отдельно вещественные и мнимые части чисел.

    
    Формально определим свойства операции сложения.
    \begin{enumerate}
        \item{ассоциативность}
        \[
            (z+w)+r = z+(w+r)
        \] 
        
        \item{коммутативность}
        \[
            z+w = w+z
        \]
        
        \item
        \[
        \forall z \in \mathbb{C} \exists (-z): z+(-z)=0
        \]
        \[
        0+z=z
        \]
        
        \item 
        \[
        Re\hspace{3px}(z+w)=Re\hspace{3px}z+Re\hspace{3px}w
        \]
        \[
        Im\hspace{3px}(z+w)=Im\hspace{3px}z+Im\hspace{3px}w
        \]
    \end{enumerate}

    \newpage

    Геометрическая интерпритация
    
    по сути своей поралельные переносы (нарисую на доске общий концепт с некоторыми буквами, зачем конкретный пример с числами. Расскажу, что по сути своей это сложение векторов (правило треугольника, паралелограмма))

    \begin{center}
    \begin{tikzpicture}[scale=1.2]
        \draw[->] (-0.5,0) -- (5,0) node[right] {$\Re$};
        \draw[->] (0,-0.5) -- (0,3) node[above] {$\Im$};
        
        % Пример 1
        \draw[->, thick, red] (0,0) -- (3,1) node[midway, below] {$3+1i$};
        \draw[->, thick, blue] (0,0) -- (1,2) node[midway, left] {$1+2i$};
        \draw[->, thick, blue, dashed] (3,1) -- (4,3);
        \draw[->, very thick, purple] (0,0) -- (4,3) node[midway, above right] {$4+3i$};
        
        \node[below right] at (3,0) {3};
        \node[left] at (0,1) {1};
        \node[below] at (1,0) {1};
        \node[left] at (0,2) {2};
        \node[below] at (4,0) {4};
        \node[left] at (0,3) {3};
    \end{tikzpicture}
    \end{center}
    
    \[
    (3 + 1i) + (1 + 2i) = (3+1) + (1+2)i = 4 + 3i
    \]

    \vspace{4cm}
    \section*{Умножение $(\mathbb{C,*})$}

    Простой пример. Возмем два компексных числа $(z=x+iy)$ и $(w=a+ib)$ и попробуем их умножить как обыкновенные вещественные числа. 
    \[
    z*w = (x+iy)(a+ib)= xa +ixb+iya+i^2yb=(xa-yb)+i(xb+ya) 
    \]
    \[
    Re\hspace{3px}(zw)=xa-yb
    \]
    \[
    Im\hspace{3px}(zw)=xb+ya
    \]
    \vspace{1cm}

    Формально определим свойства операции умножения.
    \begin{enumerate}
        \item{ассоциативность}
        \[
            (zw)r = z(wr)
        \]
        попробуем доказать с конкретными числами
        
        \item{коммутативность}
        \[
            zw = wz
        \]

        \item{дистрибутивность}
        \[
            z(w+r)=zw+zr
        \]
        попробуем доказать с конкретными числами 
        
        \item
        \[
        1*z=z
        \]
    \end{enumerate}


    \vspace{1cm}
    Чтобы представить операцию сложения геометрически на комплпексной плоскости, нужно узнать еще пару вещей, а именно что такое, сопряжение, модуль комплексного числа и аргумент комплексного числа.

    \subsection*{Сопряжение}

    операция сопряжения записывается как $\overline{z}$, где $z=x+iy$. Тогда
    \[
    \overline{z} = x-iy
    \]
    По сути просто меняем знак у мнимой части.

    Геометрически это выглядит так:
    \begin{center}
    \begin{tikzpicture}[scale=1.5, >=stealth][H]
        % Оси координат
        \draw[->, thick] (-2,0) -- (2,0) node[right] {$\text{Re}$};
        \draw[->, thick] (0,-1.5) -- (0,2) node[above] {$\text{Im}$};
        % Комплексная плоскость с сеткой (опционально)
        \draw[gray!20] (-1.8,-1.3) grid (1.8,1.8);
        % Точка z и её проекции
        \coordinate (z) at (1.2, 1.5);
        \coordinate (zbar) at (1.2, -1.5);
        % Обозначения точек
        \node[above right] at (z) {$z = a + bi$};
        \node[below right] at (zbar) {$\overline{z} = a - bi$};
        % Векторы
        \draw[ultra thick, blue, ->] (0,0) -- (z) node[midway, above left] {};
        \draw[ultra thick, red, ->] (0,0) -- (zbar) node[midway, below left] {};
        % Точки
        \fill[blue] (z) circle (2pt);
        \fill[red] (zbar) circle (2pt);
        % Проекции на оси
        \draw[dashed, gray] (z) -- (1.2, 0) node[below] {$a$};
        \draw[dashed, gray] (z) -- (0, 1.5) node[left] {$b$};
        \draw[dashed, gray] (zbar) -- (0, -1.5) node[left] {$-b$};
    \end{tikzpicture}
    \end{center}

    Свойства операции сопряжение:
    \begin{enumerate}
        \item 
        \[
        Re\hspace{3px}(\overline{z})=Re\hspace{3px}(z)
        \]
        \[
        Im\hspace{3px}(\overline{z})=-Im\hspace{3px}(z)
        \]
        \item
        \[
        \overline{(z\pm w)}=\overline{z}\pm\overline{w}
        \]
        \item 
        \[
        \overline{(zw)}=\overline{z}*\overline{w}
        \]
    \end{enumerate}

    \vspace{1cm}
    Рассмотри вот такую штуку:
    \[
    z\overline{z}=(x+iy)(x-iy)=(x^2+y^2) \in \mathbb{R}
    \]
    \{то, что результат принадлежит \mathbb{R} пригодится при определении деления\}
    
    \vspace{1cm}
    По определению \textbf{нормой} числа $z$ ($N(z)$) называется $z\overline{z}$ а модулем ($|z|$)  числа $z$ называется $\sqrt{z\overline{z}}=\sqrt{x^2+y^2}$

    \begin{tikzpicture}[scale=1.5, >=stealth]
        % Оси координат
        \draw[->, thick] (-1,0) -- (3.5,0) node[right] {$\text{Re}$};
        \draw[->, thick] (0,-1) -- (0,3) node[above] {$\text{Im}$};
        % Сетка (опционально)
        \draw[gray!15] (-0.5,-0.5) grid (3.5,2.5);
        % Комплексное число z = a + bi
        \def\a{2}
        \def\b{2}
        \coordinate (z) at (\a, \b);
        \coordinate (O) at (0,0);
        % Вектор от начала координат
        \draw[ultra thick, blue, ->] (O) -- (z) node[midway, above left] {};
        % Точка z
        \fill[blue] (z) circle (3pt) node[above right] {$z = a + bi$};
        % Проекции на оси
        \draw[dashed, gray] (z) -- (\a, 0) node[below] {$a$};
        \draw[dashed, gray] (z) -- (0, \b) node[left] {$b$};
        % Прямоугольный треугольник
        \draw[thin, blue] (O) rectangle (0.3, 0.3);
        % Модуль - гипотенуза
        \draw[red, very thick] (O) -- (z) node[midway, above, sloped] {$|z| = r$};
        % Обозначение модуля на гипотенузе
        \node[red, fill=white, inner sep=2pt] at (1, 1) {$r = \sqrt{a^2 + b^2}$};
        % Геометрическая интерпретация
        \node[align=center, font=\large] at (1.5, -1.2) 
            {Модуль $|z|$ = длина вектора от начала координат до точки $z$};
    \end{tikzpicture}


    \newpage
    \begin{table}[h]
        \centering
        \renewcommand{\arraystretch}{1.8}
        \begin{tabular}{|C|C|C|}
        \hline
        \textbf{Сложение} & \textbf{Умножение} & \textbf{Модуль} \\
        \hline
        \hline
        $(3 + 2i) + (5 - 4i)$ & $(2 + 3i) \cdot (1 - i)$ & $|3 + 4i|$ \\
        \hline
        $(-1.5 + 3i) + (2.5 - i)$ & $(-1 + 2i) \cdot (3 + 4i)$ & $|-2 - 2i|$ \\
        \hline
        $(2 + i) + (-3 + 4i) + (1 - 2i)$ & $(4 - i) \cdot (4 + i)$ & $|5|$ \\
        \hline
        $z_1 = a + bi$, $z_2 = 2a - bi$ \\
        Найдите $z_1 + z_2$ & $(1 + i)^2$ & $|-3i|$ \\
        \hline
        $(7 - \sqrt{-9}) + (2 + \sqrt{-16})$ & $(1 + i) \cdot (1 - i) \cdot (2 + 3i)$ & Докажите: $|z \cdot \overline{z}| = |z|^2$ \\
        \hline
        \end{tabular}
    \end{table}

    
\end{document}